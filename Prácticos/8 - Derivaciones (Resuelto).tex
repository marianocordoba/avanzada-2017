\documentclass[12pt]{article}
\usepackage[a4paper,margin=2cm]{geometry}
\usepackage[utf8]{inputenc}
\usepackage[spanish,es-nodecimaldot]{babel}
\usepackage{eqparbox}
\usepackage[fleqn]{amsmath}

\setlength{\mathindent}{0pt}
\newcommand{\concat}{%
  \mathbin{{+}\mspace{-8mu}{+}}%
}

\title{Practico 8 - Derivaciones}
\author{Mariano Córdoba}

\begin{document}

\begin{center}
\large \underline{\textbf{Práctico 8 - Derivaciones}}
\end{center}

\textbf{Ejercicio 1}

\begin{itemize}
    \item $ f.xs = \langle \forall_i : 0 < i < \#xs : xs.i = xs.0 \rangle $

    \bigbreak

    Casos base:\\
    $ f.[] $\\
    = \{Especificación\}\\
    $ \langle \forall_i : 0 < i < \#[] : xs.[] = xs.0 \rangle $\\
    = \{Definición de \#\}\\
    $ \langle \forall_i : 0 < i < 0 : xs.[] = xs.0 \rangle $\\
    = \{Rango vacío\}\\
    $ True $

    \bigbreak

    $ f.[x] $\\
    = \{Especificación\}\\
    $ \langle \forall_i : 0 < i < \#[x] : xs.[x] = xs.0 \rangle $\\
    = \{Definición de \#\}\\
    $ \langle \forall_i : 0 < i < 1 : xs.[x] = xs.0 \rangle $\\
    = \{Rango vacío\}\\
    $ True $

    \bigbreak

    Caso inductivo:\\
    $ f.(x \triangleright xs) $\\
    = \{Especificación\}\\
    $ \langle \forall_i : 0 < i < \#(x \triangleright xs) : (x \triangleright xs).i = (x \triangleright xs).0 \rangle $\\
    = \{Definición de \#\}\\
    $ \langle \forall_i : 0 < i < 1 + \#xs : (x \triangleright xs).i = (x \triangleright xs).0 \rangle $\\
    = \{Separación de un término\}\\
    $ (x \triangleright xs).(0 + 1) = (x \triangleright xs).0 \land \langle \forall_i : 0 < i < \#xs : (x \triangleright xs).(i + 1) = (x \triangleright xs).0 \rangle $\\
    = \{Propiedad de .\}\\
    $ xs.0 = x \land \langle \forall_i : 0 < i < \#xs : xs.i = x \rangle $\\
    = \{Leibnitz\}\\
    $ xs.0 = x \land \langle \forall_i : 0 < i < \#xs : xs.i = xs.0 \rangle $\\
    = \{Hipótesis inductiva\}\\
    $ xs.0 = x \land f.xs $
    
    \rule{\linewidth}{1px}

    \item $ f.xs.x = \langle \exists_i : 0 \le i < \#xs : xs.i = x \rangle $

    \bigbreak 

    Caso base:\\
    $ f.[].x $\\
    = \{Especificación\}\\
    $ \langle \exists_i : 0 \le i < \#[] : [].i = x \rangle $\\
    = \{Definición de \#\}\\
    $ \langle \exists_i : 0 \le i < 0 : [].i = x \rangle $\\
    = \{Rango vacío\}\\
    $ False $

    \bigbreak

    Caso inductivo:\\
    $ f.(y \triangleright xs).x $\\
    = \{Especificación\}\\
    $ \langle \exists_i : 0 \le i < \#(y \triangleright xs) : (y \triangleright xs).i = x \rangle $\\
    = \{Definición de \#\}\\
    $ \langle \exists_i : 0 \le i < 1 + \#xs : (y \triangleright xs).i = x \rangle $\\
    = \{Separación de un término\}\\
    $ (y \triangleright xs).0 = x \lor \langle \exists_i : 0 \le i < \#xs : (y \triangleright xs).(i + 1) = x \rangle $\\
    = \{Propiedad de .\}\\
    $ y = x \lor \langle \exists_i : 0 \le i < \#xs : xs.i = x \rangle $\\
    = \{Hipótesis inductiva\}\\
    $ y = x \lor f.xs.x $

    \rule{\linewidth}{1px}

    \item $ f.xs.x = \langle \forall_i : 0 \le i < \#xs : xs.i = x \rangle $

    \bigbreak

    Caso base:\\
    $ f.[].x $\\
    = \{Especificación\}\\
    $ \langle \forall_i : 0 \le i < \#[] : [].i = x \rangle $\\
    = \{Definición de \#\}\\
    $ \langle \forall_i : 0 \le i < 0 : [].i = x \rangle $\\
    = \{Rango vacío\}\\
    $ True $

    \bigbreak

    Caso inductivo:\\
    $ f.(y \triangleright xs).x $\\
    = \{Especificación\}\\
    $ \langle \forall_i : 0 \le i < \#(y \triangleright xs) : (y \triangleright xs).i = x \rangle $\\
    = \{Definición de \#\}\\
    $ \langle \forall_i : 0 \le i < 1 + \#xs : (y \triangleright xs).i = x \rangle $\\
    = \{Separación de un término\}\\
    $ (y \triangleright xs).0 = x \land \langle \forall_i : 0 \le i < \#xs : (y \triangleright xs).(i + 1) = x \rangle $\\
    = \{Propiedad de .\}\\
    $ y = x \land \langle \forall_i : 0 \le i < \#xs : xs.i = x \rangle $\\
    = \{Hipótesis inductiva\}\\
    $ y = x \land f.xs.x $

    \rule{\linewidth}{1px}

    \item $ f.xs.ys = \langle \forall_i : 0 \le i < \#xs \lor 0 \le i < \#ys : \#xs = \#ys \land xs.i = ys.i \rangle $

    \bigbreak

    Caso ($ xs = [], ys = []$):\\
    $ f.[].[] $\\
    = \{Especificación\}\\
    $ \langle \forall_i : 0 \le i < \#[] \lor 0 \le i < \#[] : \#[] = \#[] \land [].i = [].i \rangle $
    = \{Definición de \#\}\\
    $ \langle \forall_i : 0 \le i < 0 \lor 0 \le i < 0 : 0 = 0 \land [].i = [].i \rangle $
    = \{Rango vacío\}\\
    $ True $

    \bigbreak

    Caso ($ xs = [], ys = (y \triangleright ys) $):\\
    $ f.[].(y \triangleright ys) $\\
    = \{Especificación\}\\
    $ \langle \forall_i : 0 \le i < \#[] \lor 0 \le i < \#(y \triangleright ys) : \#[] = \#(y \triangleright ys) \land [].i = (y \triangleright ys).i \rangle $\\
    = \{$ \#[] \neq \#(y \triangleright ys) $\}\\
    $ \langle \forall_i : 0 \le i < \#[] \lor 0 \le i < \#(y \triangleright ys) : False \land [].i = (y \triangleright ys).i \rangle $\\
    = \{Lógica\}\\
    $ \langle \forall_i : 0 \le i < \#[] \lor 0 \le i < \#(y \triangleright ys) : False \rangle $\\
    = \{Término constante\}\\
    $ False $

    \bigbreak

    Análogamente se resuelve $ f.(x \triangleright xs).[] $

    \bigbreak

    Caso inductivo:\\
    $ f.(x \triangleright xs).(y \triangleright ys) $\\
    = \{Especificación\}\\
    $ \langle \forall_i : 0 \le i < \#(x \triangleright xs) \lor 0 \le i < \#(y \triangleright ys) : \#(x \triangleright xs) = \#(y \triangleright ys) \land (x \triangleright xs).i = (y \triangleright ys).i \rangle $\\
    = \{Definición de \#\}\\
    $ \langle \forall_i : 0 \le i < 1 + \#xs \lor 0 \le i < 1 + \#ys : 1 + \#xs = 1 + \#ys \land (x \triangleright xs).i = (y \triangleright ys).i \rangle $\\
    = \{Separación de un término, aritmética\}\\
    $ \#xs = \#ys \land (x \triangleright xs).0 = (y \triangleright ys).0 \land \langle \forall_i : 0 \le i < \#xs \lor 0 \le i < \#ys : \#xs = \#ys \land (x \triangleright xs).(i + 1) = (y \triangleright ys).(i + 1) \rangle $\\
    = \{Propiedad de .\}\\
    $ \#xs = \#ys \land x = y \land \langle \forall_i : 0 \le i < \#xs \lor 0 \le i < \#ys : \#xs = \#ys \land xs.i = ys.i \rangle $\\
    = \{Hipótesis inductiva\}\\
    $ \#xs = \#ys \land x = y \land f.xs.ys $\\

\end{itemize}

\bigbreak

\textbf{Ejercicio 2}

\begin{itemize}
    \item $ f.xs = \langle \forall_i : 0 \le i < \#xs - 1 : xs.i < xs.(i + 1) \rangle $

    \bigbreak

    Caso base:\\
    $ f.[] $\\
    = \{Especificación\}\\
    $ \langle \forall_i : 0 \le i < \#[] - 1 : [].i < [].(i + 1) \rangle $\\
    = \{Definición de \#, aritmética\}\\
    $ \langle \forall_i : 0 \le i < -1 : [].i < [].(i + 1) \rangle $\\
    = \{Rango vacío\}\\
    $ True $

    \bigbreak

    Caso inductivo:\\
    $ f.(x \triangleright xs ) $\\
    = \{Especificación\}\\
    $ \langle \forall_i : 0 \le i < \#(x \triangleright xs ) - 1 : (x \triangleright xs ).i < (x \triangleright xs ).(i + 1) \rangle $\\
    = \{Definición de \#, aritmética\}\\
    $ \langle \forall_i : 0 \le i < \#xs : (x \triangleright xs ).i < (x \triangleright xs ).(i + 1) \rangle $\\
    = \{Separación de un término\}\\
    $ (x \triangleright xs ).0 < (x \triangleright xs ).(0 + 1) \land \langle \forall_i : 0 \le i < \#xs : (x \triangleright xs ).(i + 1) < (x \triangleright xs ).(i + 2) \rangle $\\
    = \{Propiedad de .\}\\
    $ x < xs.0 \land \langle \forall_i : 0 \le i < \#xs - 1 : xs.i < xs.(i + 1) \rangle $\\
    = \{Hipótesis inductiva\}\\
    $ x < xs.0 \land f.xs $

\end{itemize}

\bigbreak

\textbf{Ejercicio 3}

\begin{itemize}
    \item $ m.xs = \langle Min_i : 0 \le i < \#xs : xs.i \rangle $

    \bigbreak

    Caso base:\\
    $ m.[] $\\
    = \{Especificación\}\\
    $ \langle Min_i : 0 \le i < \#[] : [].i \rangle $\\
    = \{Definición de \#\}\\
    $ \langle Min_i : 0 \le i < 0 : [].i \rangle $\\
    = \{Rango vacío\}\\
    $ \infty $

    \bigbreak

    Caso inductivo:\\
    $ m.(x \triangleright xs) $\\
    = \{Especificación\}\\
    $ \langle Min_i : 0 \le i < \#(x \triangleright xs) : (x \triangleright xs).i \rangle $\\ 
    = \{Definición de \#\}\\
    $ \langle Min_i : 0 \le i < 1 + \#xs : (x \triangleright xs).i \rangle $\\ 
    = \{Separación de un término\}\\
    $ (x \triangleright xs).0 `Min` \langle Min_i : 0 \le i < 1 + \#xs : (x \triangleright xs).(i + 1) \rangle $\\ 
    = \{Propiedad de .\}\\
    $ x \ Min\ \langle Min_i : 0 \le i < \#xs : xs.i \rangle $\\ 
    = \{Hipótesis inductiva\}\\
    $ x \ Min\ m.xs $

\end{itemize}

\textbf{Ejercicio 4}

\begin{itemize}
    \item $ f.xs = \langle \exists_i : 0 \le i < \#xs : xs.i = sum(xs) - xs.i \rangle $

    \bigbreak 

    Caso base:\\
    $ f.[] $\\
    = \{Especificación\}\\
    $ \langle \exists_i : 0 \le i < \#[] : [].i = sum([]) - [].i \rangle $\\
    = \{Definición de \#\}\\
    $ \langle \exists_i : 0 \le i < 0 : [].i = sum([]) - [].i \rangle $\\
    = \{Rango vacío\}\\
    $ False $

    \bigbreak

    Caso inductivo:\\
    $ f.(x \triangleright xs) $\\
    = \{Especificación\}\\
    $ \langle \exists_i : 0 \le i < \#(x \triangleright xs) : (x \triangleright xs).i = sum((x \triangleright xs)) - (x \triangleright xs).i \rangle $\\
    = \{Definición de \#\}\\
    $ \langle \exists_i : 0 \le i < 1 + \#xs : (x \triangleright xs).i = sum((x \triangleright xs)) - (x \triangleright xs).i \rangle $\\
    = \{Separación de un término\}\\
    $ (x \triangleright xs).0 = sum((x \triangleright xs)) - (x \triangleright xs).0 \lor \langle \exists_i : 0 \le i < \#xs : (x \triangleright xs).(i + 1) = sum((x \triangleright xs)) - (x \triangleright xs).(i + 1) \rangle $\\
    = \{Propiedad de .\}\\
    $ x = sum((x \triangleright xs)) - x \lor \langle \exists_i : 0 \le i < \#xs : xs.i = sum((x \triangleright xs)) - xs.i \rangle $\\
    = \{Propiedad de sum, aritmética\}\\
    $ x = sum(xs) \lor \langle \exists_i : 0 \le i < \#xs : xs.i = x + sum(xs) - xs.i \rangle $

    \bigbreak

    No se puede aplicar la hipótesis inductiva, hay que generalizar.

    \item $ g.xs.n = \langle \exists_i : 0 \le i < \#xs : xs.i = n + sum(xs) - xs.i \rangle $

    \bigbreak

    Caso base:\\
    $ g.[].n $
    = \{Especificación\}\\
    $ \langle \exists_i : 0 \le i < \#[] : [].i = n + sum([]) - [].i \rangle $\\
    = \{Definición de \#\}\\
    $ \langle \exists_i : 0 \le i < 0 : [].i = n + sum([]) - [].i \rangle $\\
    = \{Rango vacío\}\\
    $ False $

    \bigbreak

    Caso inductivo:\\
    $ f.(x \triangleright xs).n $
    = \{Especificación\}\\
    $ \langle \exists_i : 0 \le i < \#(x \triangleright xs) : (x \triangleright xs).i = n + sum((x \triangleright xs)) - (x \triangleright xs).i \rangle $\\
    = \{Definición de \#\}\\
    $ \langle \exists_i : 0 \le i < 1 + \#xs : (x \triangleright xs).i = n + sum((x \triangleright xs)) - (x \triangleright xs).i \rangle $\\
    = \{Separación de un término\}\\
    $ (x \triangleright xs).0 = n + sum((x \triangleright xs)) - (x \triangleright xs).0 \lor \langle \exists_i : 0 \le i < \#xs : (x \triangleright xs).(i + 1) = n + sum((x \triangleright xs)) - (x \triangleright xs).(i + 1) \rangle $\\
    = \{Propiedad de .\}\\
    $ x = n + sum((x \triangleright xs)) - x \lor \langle \exists_i : 0 \le i < \#xs : xs.i = n + sum((x \triangleright xs)) - xs.i \rangle $\\
    = \{Propiedad de sum, aritmética\}\\
    $ x = n + sum(xs) \lor \langle \exists_i : 0 \le i < \#xs : xs.i = x + n + sum(xs) - xs.i \rangle $\\
    = \{Asociatividad de +\}\\
    $ x = n + sum(xs) \lor \langle \exists_i : 0 \le i < \#xs : xs.i = (x + n) + sum(xs) - xs.i \rangle $\\
    = \{Hipótesis inductiva\}\\
    $ x = n + sum(xs) \lor g.xs.(x + n) $

    \bigbreak

    El resultado completo de esta derivación es:
    \[
    \begin{array}{| r @{} l }
        f ~ : ~ & [\text{Num}] \mapsto \text{Bool} \\
        \rule[-10pt]{0pt}{0pt}g ~ : ~ & \text{Num} \mapsto [\text{Num}] \mapsto \text{Bool} \\
        \cline{1-1}
        \multicolumn{2}{|l}{\rule{0pt}{12pt}\eqmakebox[lhs][r]{$f.xs$} ~ \doteq ~ g.xs.0} \\
        \multicolumn{2}{|l}{\eqmakebox[lhs][r]{$g.[].n$} ~ \doteq ~ False} \\
        \multicolumn{2}{|l}{\eqmakebox[lhs][r]{$g.(x \triangleright xs).n$} ~ \doteq ~ n + sum(xs) \lor g.xs.(x + n)}
    \end{array}
    \]

\end{itemize}

\textbf{Ejercicio 5}

\begin{itemize}
    \item $ P.xs.ys = \langle \exists_{as,bs} :: ys = as \concat xs \concat bs \rangle $

    \bigbreak

    Caso ($xs = []$):\\
    $ P.[].ys $
    = \{Especificación\}\\
    $ \langle \exists_{as,bs} :: ys = as \concat [] \concat bs \rangle $\\
    = \{Definición de$~\concat$\}\\
    $ \langle \exists_{as,bs} :: ys = as \concat bs \rangle $\\
    = \{Partición de rango ($as = [] \lor as \neq []$)\}\\
    $ \langle \exists_{as,bs} : as = [] : ys = [] \concat bs \rangle \lor \langle \exists_{as,bs} : as \neq [] : ys = as \concat bs \rangle $\\
    = \{Anidado, rango unitario\}\\
    $ \langle \exists_{bs} :: ys = [] \concat bs \rangle \lor \langle \exists_{as,bs} : as \neq [] : ys = as \concat bs \rangle $\\
    = \{Definición de$~\concat$\}\\
    $ \langle \exists_{bs} :: ys = bs \rangle \lor \langle \exists_{as,bs} : as \neq [] : ys = as \concat bs \rangle $\\
    = \{Intercambio\}\\
    $ \langle \exists_{bs} : ys = bs : True \rangle \lor \langle \exists_{as,bs} : as \neq [] : ys = as \concat bs \rangle $\\
    = \{Rango unitario\}\\
    $ True \lor \langle \exists_{as,bs} : as \neq [] : ys = as \concat bs \rangle $\\
    = \{Absorción  para el $\lor$\}\\
    $ True $

    \bigbreak

    Caso ($x \triangleright xs, ys = []$):\\
    $ P.(x \triangleright xs).[] $\\
    = \{Especificación\}\\
    $ \langle \exists_{as,bs} :: [] = as \concat (x \triangleright xs) \concat bs \rangle $\\
    = \{Igualdad de listas\}\\
    $ \langle \exists_{as,bs} :: [] = as \land [] = (x \triangleright xs) \land [] = bs \rangle $\\
    = \{Igualdad de listas\}\\
    $ \langle \exists_{as,bs} :: [] = as \land False \land [] = bs \rangle $\\
    = \{Rango vacío\}\\
    $ False $\\

    \bigbreak

    Caso ($x \triangleright xs, y \triangleright ys$):\\
    $ P.(x \triangleright xs).(y \triangleright ys) $\\
    = \{Especificación\}\\
    $ \langle \exists_{as,bs} :: (y \triangleright ys) = as \concat (x \triangleright xs) \concat bs \rangle $\\
    = \{Partición de rango ($as = [] \lor as \neq []$)\}\\
    $ \langle \exists_{as,bs} : as = [] : (y \triangleright ys) = as \concat (x \triangleright xs) \concat bs \rangle \lor \langle \exists_{as,bs} : as \neq [] : (y \triangleright ys) = as \concat (x \triangleright xs) \concat bs \rangle $\\
    = \{Anidado, rango unitario\}\\
    $ \langle \exists_{bs} :: (y \triangleright ys) = [] \concat (x \triangleright xs) \concat bs \rangle \lor \langle \exists_{as,bs} : as \neq [] : (y \triangleright ys) = as \concat (x \triangleright xs) \concat bs \rangle $\\
    = \{Reemplazo de $as \leftarrow a \triangleright as$ (válido por $as \neq []$)\}\\
    $ \langle \exists_{bs} :: (y \triangleright ys) = [] \concat (x \triangleright xs) \concat bs \rangle \lor \langle \exists_{a,as,bs} : (a \triangleright as) \neq [] : (y \triangleright ys) = (a \triangleright as) \concat (x \triangleright xs) \concat bs \rangle $\\
    = \{Definición de $\concat$, igualdad de listas ($a \triangleright as \neq [] \equiv True$)\}\\
    $ \langle \exists_{bs} :: (y \triangleright ys) = (x \triangleright xs) \concat bs \rangle \lor \langle \exists_{a,as,bs} :: (y \triangleright ys) = (a \triangleright as) \concat (x \triangleright xs) \concat bs \rangle $\\
    = \{Igualdad de listas\}\\
    $ \langle \exists_{bs} :: y = x \land ys = xs \concat bs \rangle \lor \langle \exists_{a,as,bs} :: y = a \land ys = as \concat (x \triangleright xs) \concat bs \rangle $\\
    = \{Distributiva del $\land$ con respecto a $\exists$\}\\
    $ y = x \land \langle \exists_{bs} :: ys = xs \concat bs \rangle \lor \langle \exists_{a,as,bs} :: y = a \land ys = as \concat (x \triangleright xs) \concat bs \rangle $\\
    = \{Intercambio\}\\
    $ y = x \land \langle \exists_{bs} :: ys = xs \concat bs \rangle \lor \langle \exists_{a,as,bs} : y = a : ys = as \concat (x \triangleright xs) \concat bs \rangle $\\
    = \{Anidado, rango unitario\}\\
    $ y = x \land \langle \exists_{bs} :: ys = xs \concat bs \rangle \lor \langle \exists_{as,bs} :: ys = as \concat (x \triangleright xs) \concat bs \rangle $\\
    = \{Modularización, hipótesis inductiva\}\\
    $ (y = x \land Q.xs.ys) \lor P.(x \triangleright xs).ys $

    \bigbreak

    Donde Q queda especificado por:

    $ Q.xs.ys =  \langle \exists_{bs} :: ys = xs \concat bs \rangle $

    \bigbreak

    Caso ($xs = []$):\\
    $ Q.[].ys $\\
    = \{Especificación\}\\
    $ \langle \exists_{bs} :: ys = [] \concat bs \rangle $\\
    = \{Definición de $\concat$\}\\
    $ \langle \exists_{bs} :: ys = bs \rangle $\\
    = \{Intercambio\}\\
    $ \langle \exists_{bs} : ys = bs \rangle : True $\\
    = \{Término constante\}\\
    $ True $

    \bigbreak

    Caso ($x \triangleright xs, ys = []$):\\
    $ Q.(x \triangleright xs).[] $\\
    = \{Especificación\}\\
    $ \langle \exists_{bs} :: [] = (x \triangleright xs) \concat bs \rangle $\\
    = \{Igualdad de listas\}\\
    $ \langle \exists_{bs} :: [] = (x \triangleright xs) \land [] = bs \rangle $\\
    = \{Igualdad de listas\}\\
    $ \langle \exists_{bs} :: False \land [] = bs \rangle $\\
    = \{Intercambio, rango vacío\}\\
    $ False $

    \bigbreak

    Caso ($x \triangleright xs, y \triangleright ys$)\\
    $ Q.(x \triangleright xs).(y \triangleright ys) $\\
    = \{Especificación\}\\
    $ \langle \exists_{bs} :: (y \triangleright ys) = (x \triangleright xs) \concat bs \rangle $\\
    = \{Definición de $\concat$\}\\
    $ \langle \exists_{bs} :: (y \triangleright ys) = x \triangleright (xs \concat bs) \rangle $\\
    = \{Igualdad de listas\}\\
    $ \langle \exists_{bs} :: y = x \land ys = xs \concat bs \rangle $\\
    = \{Distributiva de $\land$ con el $\exists$\}\\
    $ y = x \land \langle \exists_{bs} :: ys = xs \concat bs \rangle $\\
    = \{Hipótesis inductiva\}\\
    $ y = x \land Q.xs.ys $

    \bigbreak

    El programa completo queda entonces:
    \[
    \begin{array}{| r @{} l }
        P ~ : ~ & [\text{A}] \mapsto [\text{A}] \mapsto \text{Bool} \\
        \rule[-10pt]{0pt}{0pt}Q ~ : ~ & [\text{A}] \mapsto [\text{A}] \mapsto \text{Bool} \\
        \cline{1-1}
        \multicolumn{2}{|l}{\rule{0pt}{12pt}\eqmakebox[lhs][r]{$P.[].ys$} ~ \doteq ~ True} \\
        \multicolumn{2}{|l}{\eqmakebox[lhs][r]{$P.(x \triangleright xs).[]$} ~ \doteq ~ False} \\
        \multicolumn{2}{|l}{\eqmakebox[lhs][r]{$P.(x \triangleright xs).(y \triangleright ys)$} ~ \doteq ~ (y = x \land Q.xs.ys) \lor P.(x \triangleright xs).ys} \\
        \multicolumn{2}{|l}{\eqmakebox[lhs][r]{$Q.[].ys$} ~ \doteq ~ True} \\
        \multicolumn{2}{|l}{\eqmakebox[lhs][r]{$Q.(x \triangleright xs).[]$} ~ \doteq ~ False} \\
        \multicolumn{2}{|l}{\eqmakebox[lhs][r]{$Q.(x \triangleright xs).(y \triangleright ys)$} ~ \doteq ~ y = x \land Q.xs.ys} \\
    \end{array}
    \]
    

\end{itemize}

\textbf{Ejercicio 6}

\begin{itemize}
    \item $ P.xs = \langle\exists_{as,bs} : xs = as \concat bs : sum.as = sum.bs \rangle $

    \bigbreak

    Caso ($xs = []$):\\
    $ P.[] $\\
    = \{Especificación\}\\
    $ \langle\exists_{as,bs} : [] = as \concat bs : sum.as = sum.bs \rangle $\\
    = \{Propiedad de $\concat$\}\\
    $ \langle\exists_{as,bs} : [] = as \concat bs \land as = [] \land bs = [] : sum.as = sum.bs \rangle $\\
    = \{Anidado, rango vacío, término constante\}\\
    $ sum.[] = sum.[] $\\
    = \{Reflexividad de =\}\\
    $ True $\\

    \bigbreak

    Caso ($x \triangleright xs$):\\
    $ P.(x \triangleright xs) $\\
    = \{Especificación\}\\
    $ \langle\exists_{as,bs} : (x \triangleright xs) = as \concat bs : sum.as = sum.bs \rangle $\\
    = \{Partición de rango ($as = [] \lor as \neq []$)\}\\
    $ \langle\exists_{as,bs} : (x \triangleright xs) = as \concat bs \land as = [] : sum.as = sum.bs \rangle \lor \langle\exists_{as,bs} : (x \triangleright xs) = as \concat bs : sum.as = sum.bs \rangle $\\
    = \{Anidado, rango unitario en el primer término.\\\hspace*{18pt}Reemplazo de $as \leftarrow a \triangleright as$ en el segundo término (válido por $as \neq []$)\}\\
    $ \langle\exists_{bs} : (x \triangleright xs) = [] \concat bs : sum.[] = sum.bs \rangle \lor \langle\exists_{a,as,bs} : (x \triangleright xs) = (a \triangleright as) \concat bs : sum.(a \triangleright as) = sum.bs \rangle $\\
    = \{Propiedad de $\concat$, propiedad de $sum$, igualdad de listas\}\\
    $ \langle\exists_{bs} : (x \triangleright xs) = bs : 0 = sum.bs \rangle \lor \langle\exists_{a,as,bs} : x = a \land xs = as \concat bs : sum.(a \triangleright as) = sum.bs \rangle $\\
    = \{Rango unitario, propiedad de $sum$ en el primer término.\\\hspace*{18pt}Anidado en el segundo término\}\\
    $ x + sum.xs = 0 \lor \langle\exists_a : x = a : \langle\exists_{as,bs} : xs = as \concat bs : sum.(a \triangleright as) = sum.bs \rangle\rangle $\\
    = \{Rango unitario, propiedad de $sum$\}\\
    $ x + sum.xs = 0 \lor \langle\exists_{as,bs} : xs = as \concat bs : x + sum.as = sum.bs \rangle $

    \bigbreak

    No se puede aplicar la hipótesis inductiva, hay que generalizar.

    \bigbreak

    \item $ Q.n.xs = \langle\exists_{as,bs} : xs = as \concat bs : n + sum.as = sum.bs \rangle$

    \bigbreak

    Caso ($xs = []$):\\
    $ Q.n.[] $\\
    = \{Especificación\}\\
    $ \langle\exists_{as,bs} : [] = as \concat bs : n + sum.as = sum.bs \rangle $\\
    = \{Propiedad de $\concat$\}\\
    $ \langle\exists_{as,bs} : [] = as \concat bs \land as = [] \land bs = [] : n + sum.as = sum.bs \rangle $\\
    = \{Anidado, rango vacío, término constante\}\\
    $ n + sum.[] = sum.[] $\\
    = \{Definición de $sum$, aritmética\}\\
    $ n = 0 $\\

    \bigbreak

    Caso ($x \triangleright xs$):\\
    $ Q.n.(x \triangleright xs) $\\
    = \{Especificación\}\\
    $ \langle\exists_{as,bs} : (x \triangleright xs) = as \concat bs : n + sum.as = sum.bs \rangle $\\
    = \{Partición de rango ($as = [] \lor as \neq []$)\}\\
    $ \langle\exists_{as,bs} : (x \triangleright xs) = as \concat bs \land as = [] : n + sum.as = sum.bs \rangle \lor \langle\exists_{as,bs} : (x \triangleright xs) = as \concat bs : n + sum.as = sum.bs \rangle $\\
    = \{Anidado, rango unitario en el primer término.\\\hspace*{18pt}Reemplazo de $as \leftarrow a \triangleright as$ en el segundo término (válido por $as \neq []$)\}\\
    $ \langle\exists_{bs} : (x \triangleright xs) = [] \concat bs : n + sum.[] = sum.bs \rangle \lor \langle\exists_{a,as,bs} : (x \triangleright xs) = (a \triangleright as) \concat bs : n + sum.(a \triangleright as) = sum.bs \rangle $\\
    = \{Propiedad de $\concat$, propiedad de $sum$, aritmética, igualdad de listas\}\\
    $ \langle\exists_{bs} : (x \triangleright xs) = bs : n = sum.bs \rangle \lor \langle\exists_{a,as,bs} : x = a \land xs = as \concat bs : n + sum.(a \triangleright as) = sum.bs \rangle $\\
    = \{Rango unitario, propiedad de $sum$ en el primer término.\\\hspace*{18pt}Anidado en el segundo término\}\\
    $ x + sum.xs = n \lor \langle\exists_a : x = a : \langle\exists_{as,bs} : xs = as \concat bs : n + sum.(a \triangleright as) = sum.bs \rangle\rangle $\\
    = \{Rango unitario, propiedad de $sum$\}\\
    $ x + sum.xs = n \lor \langle\exists_{as,bs} : xs = as \concat bs : n + x + sum.as = sum.bs \rangle $\\
    = \{Asociatividad de $+$\}\\
    $ x + sum.xs = n \lor \langle\exists_{as,bs} : xs = as \concat bs : (n + x) + sum.as = sum.bs \rangle $\\
    = \{Hipótesis inductiva\}\\
    $ x + sum.xs = n \lor Q.(n + x).xs $\\

    El resultado completo de esta derivación es:
    \[
    \begin{array}{| r @{} l }
        P ~ : ~ & [\text{Num}] \mapsto \text{Bool} \\
        \rule[-10pt]{0pt}{0pt}Q ~ : ~ & \text{Num} \mapsto [\text{Num}] \mapsto \text{Bool} \\
        \cline{1-1}
        \multicolumn{2}{|l}{\rule{0pt}{12pt}\eqmakebox[lhs][r]{$P.xs$} ~ \doteq ~ Q.0.xs} \\
        \multicolumn{2}{|l}{\eqmakebox[lhs][r]{$Q.n.xs$} ~ \doteq ~ n = 0} \\
        \multicolumn{2}{|l}{\eqmakebox[lhs][r]{$Q.n.(x \triangleright xs)$} ~ \doteq ~ x + sum.xs = n \lor Q.(x + n).xs}
    \end{array}
    \]


\end{itemize}

\textbf{Ejercicio 7}

\begin{itemize}
    \item $ P.xs.ys  = \langle Min_{i,j} : 0 \le i < \#xs \land 0 \le j < \#ys : | xs.i - ys.j | \rangle $

    \bigbreak

    Caso ($xs = []$):\\
    $ P.[].ys $\\
    = \{Especificación\}\\
    $ \langle Min_{i,j} : 0 \le i < \#[] \land 0 \le j < \#ys : | [].i - ys.j | \rangle $\\
    = \{Anidado\}\\
    $ \langle Min_i : 0 \le i < \#[] : \langle Min_j : 0 \le j < \#ys : | [].i - ys.j | \rangle\rangle $\\
    = \{Definición de \#\}\\
    $ \langle Min_i : 0 \le i < 0 : \langle Min_j : 0 \le j < \#ys : | [].i - ys.j | \rangle\rangle $\\
    = \{Rango vacío\}\\
    $ +\infty $

    Análogamente se resuelve $ P.xs.[] $

    \bigbreak

    Caso ($x \triangleright xs, y \triangleright ys$):\\
    $ P.(x \triangleright xs).(y \triangleright ys) $\\
    = \{Especificación\}\\
    $ \langle Min_{i,j} : 0 \le i < \#(x \triangleright xs) \land 0 \le j < \#(y \triangleright ys) : | (x \triangleright xs).i - (y \triangleright ys).j | \rangle $\\
    = \{Definición de \#\}\\
    $ \langle Min_{i,j} : 0 \le i < 1 + \#xs \land 0 \le j < 1 + \#ys : | (x \triangleright xs).i - (y \triangleright ys).j | \rangle $\\
    = \{Separación de un término, cambio de variable\}\\
    $ | (x \triangleright xs).0 - (y \triangleright ys).0 | ~`Min`~ \langle Min_{i,j} : 0 \le i < \#xs \land 0 \le j < \#ys : | (x \triangleright xs).(i + 1) - (y \triangleright ys).(j + 1) | \rangle $\\
    = \{Propiedad de .\}\\
    $ | x - y | ~`Min`~ \langle Min_{i,j} : 0 \le i < \#xs \land 0 \le j < \#ys : | xs.i - ys.j | \rangle $\\
    = \{Hipótesis inductiva\}\\
    $ | x - y | ~`Min`~ P.xs.ys $\\

\end{itemize}

\textbf{Ejercicio 8}

\begin{itemize}
    \item $pares.xs = \langle N_i : 0 \le i < \#xs : par.(xs.i) \rangle$

    Caso base:\\
    $ pares.[] $\\
    = \{Especificación\}\\
    $ \langle N_i : 0 \le i < \#[] : par.([].i) \rangle $\\
    = \{Definición de \#, rango vacío\}\\
    $ 0 $

    \bigbreak

    Caso inductivo:\\
    $ pares.(x \triangleright xs) $\\
    = \{Especificación\}\\
    $ \langle N_i : 0 \le i < \#(x \triangleright xs) : par.((x \triangleright xs).i) \rangle $\\
    = \{Definición de \#\}\\
    $ \langle N_i : 0 \le i < 1 + \#xs : par.((x \triangleright xs).i) \rangle $\\
    = \{Separación de un término, cambio de variable\}\\
    $ \langle N :: par.((x \triangleright xs).0) \rangle + \langle N_i : 0 \le i < \#xs : par.((x \triangleright xs).(i + 1)) \rangle $\\
    = \{Propiedad de .\}\\
    $ \langle N :: par.(x) \rangle + \langle N_i : 0 \le i < \#xs : par.(xs.i) \rangle $\\
    = \{Hipótesis inductiva\}\\
    $ \langle N :: par.(x) \rangle + pares.xs $\\

    El primer término es 1 o 0 dependiendo si x es par o no, es decir, tenemos:
    \[
    \begin{array}{| r @{} l }
        \rule[-10pt]{0pt}{0pt}pares ~ : ~ & [\text{Num}] \mapsto \text{Num} \\
        \cline{1-1}
        \multicolumn{2}{|l}{\rule{0pt}{12pt}\eqmakebox[lhs][r]{$pares.[]$} ~ \doteq ~ 0} \\
        \multicolumn{2}{|l}{\eqmakebox[lhs][r]{$pares.((2n) \triangleright xs)$} ~ \doteq ~ 1 + pares.xs} \\
        \multicolumn{2}{|l}{\eqmakebox[lhs][r]{$pares.((2n + 1) \triangleright xs)$} ~ \doteq ~ pares.xs}
    \end{array}
    \]

    Análogamente podemos derivar $impares.xs = \langle N_i : 0 \le i < \#xs : impar.(xs.i) \rangle$

    \bigbreak

    Como queremos recorrer la lista una sola vez, necesitamos usar tuplas.

    \item $f.xs = (pares.xs, impares.xs)$

    \bigbreak

    Caso ($xs = []$):\\
    $f.[]$\\
    = \{Especificación\}\\
    $(pares.[], impares.[])$\\
    = \{Definición de $pares$ e $impares$\}\\
    $(0, 0)$

    \bigbreak

    Caso ($2n \triangleright xs$):\\
    $f.(2n \triangleright xs)$\\
    = \{Especificación\}\\
    $(pares.(2n \triangleright xs), impares.(2n \triangleright xs))$\\
    = \{Definición de $pares$ e $impares$\}\\
    $(1 + pares.xs, impares.xs)$\\
    = \{Introducción de $a$ y $b$ como definiciones locales\}\\
    $(1 + a, b)$\\
    $[\![a = pares.xs, b = impares.xs ]\!]$\\
    = \{Igualdad de pares\}\\
    $(1 + a, b)$\\
    $[\![(a, b) = (pares.xs, impares.xs) ]\!]$\\
    = \{Hipótesis inductiva\}\\
    $(1 + a, b)$\\
    $[\![(a, b) = f.xs ]\!]$

    \bigbreak

    El caso $(2n + 1) \triangleright xs$ se resuelve de la misma manera.

    \bigbreak

    Así, la función $f$ queda:
    \[
    \begin{array}{| r @{} l }
        \rule[-10pt]{0pt}{0pt}f ~ : ~ & [\text{Num}] \mapsto \text{(Num, Num)} \\
        \cline{1-1}
        \multicolumn{2}{|l}{\rule{0pt}{12pt}\eqmakebox[lhs][r]{$f.[]$} ~ \doteq ~ (0, 0)} \\
        \multicolumn{2}{|l}{\eqmakebox[lhs][r]{$f.((2n) \triangleright xs)$} ~ \doteq ~ (1 + a, b)} \\
        \multicolumn{2}{|l}{\eqmakebox[lhs][r]{$f.((2n + 1) \triangleright xs)$} ~ \doteq ~ (a, 1 + b)} \\
        \multicolumn{2}{|l}{\eqmakebox[lhs][r]{} ~  ~ \hspace*{15px}[\![(a, b) = f.xs]\!]}
    \end{array}
    \]

\end{itemize}

\end{document}